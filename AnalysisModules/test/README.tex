\documentclass[english]{article}
\usepackage[T1]{fontenc}
\usepackage[latin1]{inputenc}
\usepackage{listings}
\usepackage[a4paper]{geometry}
\geometry{top=1.0in,bottom=1.0in,left=1.0in,right=1.0in}

\begin{document}

\title{README instructions for analysis}
\author{P. Dudero}
\date{April 2011}
\maketitle

\lstset{basicstyle=\small}

%% ======================================================================
\section{2010 analysis procedures}
%% ======================================================================

%%--------------------------------------------------
\subsection{Storage}
\begin{itemize}
\item Due to its superior performance all samples were migrated to hadoop
(/hdfs/cms/...) as much as possible. Phedexed samples were migrated to
/hdfs/cms/phedex by Jeremy, whereas mu skims of data and large background
samples were put under /hdfs/cms/skim/mu, and signal samples were put under
/hdfs/cms/user/heavynu/HeavyNuRecoFromHLT/38X.
\item Analysis output was stored in various directories under
\verb#/local/cms/user/dudero#. Typically the directory was named after
the configuration file that was run, with ``\_cfg.py'' stripped from
the name.  Lots of other analysis flotsam and jetsam, some worthwhile
and some ancient history, can probably also be found under these
directories...
\end{itemize}

%%--------------------------------------------------
\subsection{MC Scaling to Lumi}
Subsequent sections of this README presume that MC samples are scaled to
some luminosity prior to execution of subsequent steps. In fact, historically
the MC samples were always scaled immediately after completion of execution.
This strategy may be revisited and altered as necessary; in any case the
raw MC samples were always kept around as well, as backup.

The multiscale.C script is checked in to CVS and provides capability to
scale all the histograms in many files at once. It takes a text file
appropriately formatted, as well as other formal parameters as shown
below.

The downside of this implementation is the number of times one has to rescale,
since every study indicated below leads to a new set of files, the number
of ``files2scale.txt'' files themselves proliferated.

\subsubsection{Preparation}
\begin{enumerate}
\item edit a ``files2scale.txt'' file with the appropriate information
(see the one checked in to CVS already for signal, ``sigfiles2scale.txt'').
\end{enumerate}
\subsubsection{Synopsis}
\begin{lstlisting}
root [0] .L multiscale.C+
root [1] multiscale("files2scale.txt",36.1,"NLO")
\end{lstlisting}

%%--------------------------------------------------
\subsection{Signature}
\begin{lstlisting}
void multiscale(const char* filewithpaths,
		float integluminvpb,
		const char* csordersuffix="",
		bool writeErrors=false)
\end{lstlisting}

%%--------------------------------------------------
\subsection{Formal Parameter Inputs}
\begin{tabular}{|l|l|}
\hline
filewithpaths & space or tab-separated columns with filename, cross-section,  \\
              & number of events, and optional k-factors                      \\ \hline
integluminvpb & pretty self explanatory, what lumi to scale to                \\ \hline
csordersuffix & suffix to affix to the output files, e.g. ``NLO'', ``Meas'', etc. \\ \hline
writeErrors   & optional argument to turn on errors (if they haven't been already) \\
              & and an attempt at handling them correctly during scaling      \\ \hline
\end{tabular}

%% ======================================================================
\section{Z+Jets normalization}
%% ======================================================================

%%--------------------------------------------------
\subsection{Preparation}
\begin{enumerate}
\item Make data and Z+Jets MC histograms available
\item Create the ``other backgrounds'' root file containing the sum of backgrounds to be considered
   (both ttbar and VV are minimally required). This is currently done with a superPlot config file:
\begin{lstlisting}
root [0] .L superPlot.C+
root [1] superPlot("addback2zpeak.cfg")
\end{lstlisting}
\item CAREFULLY adjust the parameters defined below in myChi2Fit.C
\end{enumerate}

%%--------------------------------------------------
\subsection{Synopsis}
\begin{lstlisting}
root [0] .L myChi2Fit.C+
root [1] myChi2Fit()
\end{lstlisting}

%%--------------------------------------------------
\subsection{Signature}
\begin{lstlisting}
void myChi2Fit(double ctr_xsecpb=1.12,
	       double xsecincpb=0.01,
	       const string& refpath="data.root:hNu/cut6_Mu1HighPt/mMuMuZoom",
	       const string& path2scale="fall10zjets.root:hNu/cut6_Mu1HighPt/mMuMuZoom",
	       const string& otherbckgrnd="zpeakbackgrnd.root:sumback_mMuMuZoom_cut6",
	       double minMLL=72.0,
	       double maxMLL=112.0)
\end{lstlisting}

%%--------------------------------------------------
\subsection{Formal Parameter Inputs}
\begin{tabular}{|l|l|}
\hline
ctr\_xsecpb   & where the multiplier value is expected; the scan is centered here \\
              &  (the scan was historically done in xsec space)                   \\ \hline
xsecincpb     & how much to increment the scan by for each iteration              \\ \hline
refpath       & the histogram path for the "reference" (data) histogram           \\ \hline
path2scale    & path to the histogram to be scaled (MC)                           \\ \hline
otherbckgrnd  & path to the histogram containing "other background"               \\ \hline
minMLL,maxMLL & defines the Z-peak window over which to do the normalization      \\ \hline
\end{tabular}

%%--------------------------------------------------
\subsection{Global Parameter Inputs}
\begin{tabular}{|l|l|}
\hline
h1sf,h2presf,h3sf   & prescale factors for reference (perhaps another MC), MC and other \\ \hline
rebinx              & rebinning on the x-axis to boost per-bin statistics, 0=none       \\ \hline
luminvpb,nevents    & required for doing the scan in xsec space; can be set trivially to \\ 
                    & accomplish a unitless normalization                                \\ \hline
\end{tabular}

%% ======================================================================
\section{Trigger Efficiency studies}
%% ======================================================================

%%--------------------------------------------------
\subsection{On Data}
\subsubsection{Preparation}
\begin{enumerate}
\item All data samples available
\item Configure trigger matching on data samples with proper parameters \\
 (\verb#HeavyNu/AnalysisModules/python/hnutrigmatch_cfi.py#)
\end{enumerate}

\subsubsection{Synopsis}
\begin{enumerate}
%
\item \verb#cmsRun heavyNuAnalysis_cfg.py# on data, trigger efficiency is collected in folders
MuTightInZwin, Mu1TrigMatchesInZwin, Mu2TrigMatchesInZwin, Mu1Mu2TrigMatchesInZwin
%
\item use superPlot to generate plots, print trigger eff. bin info in xterm window
(contents of \verb#~/.rootrc# contains path to superPlot source:
\begin{lstlisting}
ACLiC.IncludePaths:     -Imypath -I$(ROOTSYS)/include
\end{lstlisting}
\begin{lstlisting}
root [0] .L superPlot.C+
root [1] superPlot("hnutrigEffAsymOfficial.cfg")
\end{lstlisting}
%
\end{enumerate}

%%--------------------------------------------------
\subsection{On MC background}
\subsubsection{Preparation}
\begin{enumerate}
%
\item All MC background samples available and scaled to the appropriate lumi
%
\item Adjust code in HeavyNuTrigger::simulateForMC with new bin info from run on data
\end{enumerate}

\subsubsection{Synopsis}
\begin{enumerate}
%
\item \verb#cmsRun hnuTrigEffStudy_cfg.py# on all MC samples
%
\item Use superPlot to print final statistics (also scales Z histos according to Z norm)
\begin{lstlisting}
root [0] .L superPlot.C+
root [1] superPlot("finalstatsTrigEffback.cfg",false,false) # suppresses diagnostics
\end{lstlisting}
%
\item Calculate manually from the numbers printed what the up/down fluctuation is.
\end{enumerate}

%%--------------------------------------------------
\subsection{On MC signal}

Same as for background except as noted. Given the number of signal
files, some more automation is needed. \verb#spDriver.C# is a script
that invokes \verb#superPlot# a number of times using a text file as
input to configure variables such as filename.  It takes the input
configuration script (which presumably contains unresolved aliii -
plural of alias in my book) and for each line in the ``.drv'' file it
writes a new fully resolved config file to the {\tt /tmp} directory
and executes \verb#superPlot# on that temporary config file.
%
\begin{enumerate}
%
\item Run spDriver:
\begin{lstlisting}
root [0] .L spDriver.C+
root [1] spDriver("finalstatsTrigEffSig.cfg","signalfiles.drv",false,false)
\end{lstlisting}
%
\item To save the output to a file, ROOT syntax allows 
\verb# root[0] myscript(); >myoutputfile.txt#
%
\item Throw the text file into a spreadsheet - since the text here is
formatted for human readability and not spreadsheet parseability, one
can overcome the problem by including the following characters besides
space, tab and comma, as ``text csv'' separators: (\%) and set ``merge
delimiters'' to true (oocalc).  An example ``.ods'' spreadsheet
(\verb#finalstats_trigeffunc_signal.ods#) has been checked in to CVS
(with subsequent formatting) for perusal.
%
\end{enumerate}

%% ======================================================================
\section{Muon ID/Isolation Efficiency studies}
%% ======================================================================

In contrast to the trigger studies, these are comparison studies between
MC and data, so the same steps apply to both initially:

%%--------------------------------------------------
\subsection{Preparation}
%
\begin{enumerate}
\item Data and Z+Jets MC samples available
\end{enumerate}

%%--------------------------------------------------
\subsection{Synopsis}
%
\begin{enumerate}
%
\item \verb#cmsRun hnuMuSelectEffStudy_cfg.py# on data and Z+Jets MC
samples, muon loose and tight selection and isolation efficiencies are
collected in folders Mu1tagInZwin, Mu2tagInZwin,
Mu1tagMu2passesLooseInZwin, Mu2tagMu1passesLooseInZwin,
Mu1passesTightInZwin, Mu2passesTightInZwin, Mu1Mu2passesTightInZwin.
%
\item use superPlot to generate plots
%
\begin{lstlisting}
root [0] .L superPlot.C+
root [1] superPlot("hnuMuSelectEffAsymOfficial.cfg")
\end{lstlisting}
%
\item Edit hnuMuSelectEffAsymOfficial.cfg to switch between loose and
tight plots (comment/uncomment aliii at the top), repeat previous
step.
%
\item Another ``.cfg'' file is in CVS that shows more detail,
 ``hnuMuSelectEffAsymDetailed.cfg''
\end{enumerate}

Since the last study performed on 2010 data showed no substantial
difference between MC and data, no further computation was required,
although the machinery for it is commented out in the python config
file, and is probably buried in CVS history for the source.

%% ======================================================================
\section{JES/MES studies}
%% ======================================================================

%%--------------------------------------------------
\subsection{Preparation}
%
\begin{enumerate}
\item All scaled MC samples available
\end{enumerate}

%%--------------------------------------------------
\subsection{Synopsis}
%
\begin{enumerate}
%
\item \verb#cmsRun hnuJESMESstudy_cfg.py# on all samples,
JES and MES fluctuated copies of all analysis histograms are
collected in folders hNuJEShi,hNuJESlo,hNuMEShi,hNuMESlo
%
\item For all MC background EXCEPT Z+Jets, use superPlot to generate
tables of numbers for background
%
\begin{lstlisting}
root [0] .L superPlot.C+
root [1] superPlot("finalstatsJESMESback.cfg",false,false); >jesmesback.txt
\end{lstlisting}
%
\item For all Z+Jets MC background, use residualJESMES.C to generate
tables of numbers for background - read the comments in the source
file, probably the best documented script for this whole analysis.
Maybe that's not saying as much as I would like...
%
\begin{lstlisting}
dudero@zebra02 test]$ root fall10zjets.root
root [0] 
Attaching file fall10zjets.root as _file0...
root [1] .x residualJESMES(_file0)
root [2] .x residualJESMES(_file0,"MES")
\end{lstlisting}
%
\item use spDriver to generate tables of numbers for signal
%
\begin{lstlisting}
root [0] .L spDriver.C+
root [1] spDriver("finalstatsJESMESsig.cfg","signalfiles.drv",false,false); >jesmessig.txt
\end{lstlisting}
%
\end{enumerate}

Often I'll put these tables into a spreadsheet to help with final
computation. See notes on importing saved text output into oocalc under
trigger efficiency studies, and see ``jesmessig.ods'' and ``jesmesback.ods''
in CVS.

%% ======================================================================
\section{Cut Optimization}
%% ======================================================================

%%--------------------------------------------------
\subsection{Preparation}
%
\begin{enumerate}
\item run ``savehistos.C'', which extracts histograms from all of the
background and signal MC samples and saves them to a local file.
%
\begin{lstlisting}
root [0] .L savehistos.C+
root [1] savehistos("files2optim.list","optimMWR.root")
\end{lstlisting}
%
File ``files2optim.list'' contains the list of histograms to pull
and a second column that indicates shortened unique names to give them
in the new flat root file (otherwise there would be over a hundred
versions of the same histogram ``mwr''); ``optimMWR.root'' is the
name of the output file to be generated.
%
\end{enumerate}

%%--------------------------------------------------
\subsection{Synopsis}
%
To scan for optimal MWR cut values,
\begin{lstlisting}
root [0] .L opticut.C+
root [1] opticut(0) # mode 0 => saves bkgrnd fit parameters to a ".h" file
root [2] opticut(1) # mode 1 => scans MWR cut values, saves results to "mwrscantree.root"
root [3] opticut(2) # mode 2 => reads tree, prints results table & tons of plots
\end{lstlisting}
%
To scan for optimal MLL cut values, modes 3,4, and 5 do the same thing
for MLL (must have optimMLL.root prepared).

%% ======================================================================
\section{Final Statistics, Background}
%% ======================================================================

%%--------------------------------------------------
\subsection{Cut Flow Table}

\subsubsection{Synopsis}
%
To get a cut flow table for background, run spDriver as follows:
%
\begin{lstlisting}
root [0] .L spDriver.C+
root [1] spDriver("cuttable.cfg","cuttable.drv",false,false)
\end{lstlisting}

This now produces latex-formatted output that can be captured as
described above and then inserted directly into a tabular environment
within a latex document. Some further editing is still required to
add total background numbers, errors, etc.

%%--------------------------------------------------
\subsection{Systematics and total errors, Background}

Currently it is necessary to determine total systematics
for background process by using the ``spreadsheet of power''.
This handles correlated and uncorrelated errors correctly,
and totals uncorrelated errors in quadrature. A number of
rows at the top of the spreadsheet are reserved for cutflow
numbers and final numbers after optimization. Any of these
rows can be copied into the top row, which is used to calculate
the total error for the final number on the far right. The
error shows up in red at the bottom right.

%%--------------------------------------------------
\subsection{Final numbers with optimized cuts, Background}

\subsubsection{Preparation}
%
\begin{enumerate}
\item Run Cut Optimization, which generates information necessary to
decide what the optimal cuts (MLL, MWR) should be for the analysis.
This will also produce files in preparation for use in this step.
\end{enumerate}

%% ======================================================================
\section{Final Statistics, Signal}
%% ======================================================================

One ROOT script puts out all final yields, systematic errors, and total
errors, once each for every mass point hypothesis.:


%%--------------------------------------------------
\subsection{Preparation}

\begin{itemize}
\item File ``optimMWR.root'' must be ready to go, as described above.
\item Make sure the optimized cut values and total ``other'' uncertainty
numbers at the top of the script are as you want them to be.
\end{itemize}

%%--------------------------------------------------
\subsection{Synopsis}
%
\begin{lstlisting}
root [0] .L sigtable.C+
root [1] sigtable()
\end{lstlisting}

This file spits out a table that currently looks like this:
%
\begin{verbatim}
MWR   MNu  Cut     Ssig   Serr  JES(%) MES(%) MC(%) Tot(%)
==============================================================
 700  100  560 |  16.58   1.86   9.2%   1.3%   3.3%  11.2%
 700  200  560 |  44.16   3.70   6.1%   0.7%   1.9%   8.4%
 700  300  560 |  48.66   3.36   3.8%   1.0%   1.6%   6.9%
 700  400  560 |  36.50   2.47   3.6%   0.9%   1.6%   6.8%
 700  500  560 |  19.79   1.31   3.3%   1.1%   1.6%   6.6%
 700  600  560 |   4.66   0.32   2.9%   2.2%   1.9%   6.8%
==============================================================
 800  100  640 |   7.79   0.81   8.0%   2.1%   3.5%  10.5%
 800  200  640 |  26.62   1.94   4.5%   0.6%   1.8%   7.3%
 800  300  640 |  30.65   2.04   3.4%   0.7%   1.6%   6.6%
...
...
...
\end{verbatim}

where Ssig = total signal yield, Serr = total signal error;
the subsequent columns break down the percentage uncertainties.

\end{document}
